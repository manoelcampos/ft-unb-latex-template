\documentclass[a4paper,12pt,openright,titlepage,oneside]{book}

%\usepackage[english,brazil]{babel} 
%Define regra de gramática para separar síbalas {babel}
%e altera os títulos (como chapters, sections, references) para português
\usepackage[brazil]{babel}

% Define uso de caracteres acentuados no PDF gerado
% e permite copiar corretamente o texto do PDF.
\usepackage[T1]{fontenc}
\usepackage[utf8]{inputenc}

% Adição do pacote do template da UnB
\usepackage{template-FT-UnB/ft2unb}

\DeclareGraphicsExtensions{.jpg,.pdf,.mps,.png,.gif, .eps}
\graphicspath{imagens} %define diretório de imagens

%Arquivo com lista com hifenização correta de algumas palavras.
%Defina novas palavras no arquivo a medida que verificar que a hifenização automática
%etá errada para tais palavras.
%----------------DEFINE FORMA CORRETA DE HIFENIZAÇÃO PARA ALGUMAS PALAVRAS ----------------
\hyphenation{cons-tru-a}
\hyphenation{e-xem-plo}
\hyphenation{e-xem-plar}
\hyphenation{e-le-men-to}
\hyphenation{e-le-men-tar}
\hyphenation{ma-nu-al}
\hyphenation{res-pos-ta}
\hyphenation{ca-rac-te-ris-ti-ca}
\hyphenation{ca-rac-te-ris-ti-cas}
\hyphenation{ca-rac-te-ris-ti-co}
\hyphenation{ca-rac-te-ris-ti-cos}
\hyphenation{cor-res-pon-den-ci-as}
\hyphenation{cons-tru-am}
\hyphenation{re-a-li-za-da}
\hyphenation{re-a-li-za-do}
\hyphenation{re-a-li-za-das}
\hyphenation{re-a-li-za-dos}
\hyphenation{i-ne-xis-ten-cia}
\hyphenation{i-ne-xis-te}
\hyphenation{e-xis-te}
\hyphenation{di-fe-ren-te}
\hyphenation{di-fe-ren-tes}
\hyphenation{dei-xan-do}
\hyphenation{ins-ta-la-do} 
\hyphenation{ins-ta-la-dos} 
\hyphenation{ins-ta-la-da}
\hyphenation{ins-ta-la-das}
\hyphenation{re-gis-tra-do}
\hyphenation{re-gis-tra-dos}
\hyphenation{re-gis-tra-da}
\hyphenation{re-gis-tra-das}
\hyphenation{des-cre-ve}
\hyphenation{res-pei-to}
\hyphenation{re-a-li-za}
\hyphenation{re-a-li-zar}
\hyphenation{a-tu-a-li-zar}
\hyphenation{a-tu-a-li-zan-do}
\hyphenation{a-tu-a-li-za-do}
\hyphenation{a-tu-a-li-za-da}
\hyphenation{fun-ci-o-na-li-da-de}
\hyphenation{pos-si-bi-li-da-de}
\hyphenation{dis-po-si-ti-vo}
\hyphenation{dis-po-si-ti-vos}
\hyphenation{e-xis-te}
\hyphenation{e-xis-tir}
\hyphenation{des-co-ber-ta}
\hyphenation{des-co-ber-to}
\hyphenation{de-sig-nar}
\hyphenation{de-sig-na-do}
\hyphenation{o-pe-ra-ci-o-nais} 
\hyphenation{o-pe-ra-ci-o-nal}
\hyphenation{con-si-de-ra-do}
\hyphenation{con-si-de-ra-dos}
\hyphenation{ou-tro}
\hyphenation{ou-tra}
\hyphenation{ou-tros}
\hyphenation{ou-tras}
\hyphenation{e-xis-ten-te}
\hyphenation{e-xis-ten-tes}
\hyphenation{LuaOnTV}
\hyphenation{De-fi-ni-tion}
%------------------------------------------------------------------------------------------ 

% ALTERE OS VALORES DENTRO DAS CHAVES DOS COMANDOS NESTA SEÇÃO PARA INCLUIR OS SEUS DADOS E DADOS DA SUA
% DISSERTAÇÃO DE MESTRADO OU TESE DE DOUTORADO
% -----------------------------------------------------------------------------------------------------

%\onehalfspacing
\title{TÍTULO COMPLETO DA SUA DISSERTAÇÃO/TESE}
\author{SEU NOME COMPLETO}
\date{2011-03-16} %data da defesa

\grau{Mestre} %Mestre ou Doutor
\area{Engenharia Elétrica} %Nome do curso
\siglaarea{ENE} %Sigla do departamento
\tipodemonografia{Dissertação} %Dissertação ou Tese
\programa{Mestrado} %Mestrado ou Doutorado
\autorendereco{SEU ENDEREÇO COMPLETO AQUI.} %Endereço do autor da dissertação/tese
\totalpgs{147} %total de páginas atualmente na sua dissertação
\dia{16} %dia da defesa
\mes{Junho} %mês da defesa
\ano{2011} %ano da defesa
\numpublicacao{xxx/AAAA} %número da publicação (após a defesa, tal número deve ser obtido na secretaria)

%PPGENE.DM  = Programa de Pós Graduação em ENgenharia Elétrica.Dissertação de Mestrado
%PPGENE.TD  = Programa de Pós Graduação em ENgenharia Elétrica.Tese de Doutorado
\siglapublicacao{PPGENE.DM}

\titulolinhai{TÍTULO DA SUA}
\titulolinhaii{DISSERTAÇÃO DE MESTRADO}
\titulolinhaiii{OU TESE DE DOUTORADO}
\titulolinhaiv{}

\autori{SEU NOME COMPLETO AQUI}
%Caso seu nome não caiba em uma única linha, divida ele nos comandos abaixo
%\autorii{} 
%\autoriii{}

\membrodabancai{Prof. Dr. NOME DO SEU ORIENTADOR, ENE/UnB}
\membrodabancaifuncao{Orientador}
\membrodabancaii{Prof. Fulano de Tal 2, ENE/UnB}
\membrodabancaiifuncao{Examinador interno}
\membrodabancaiii{Prof. Fulano de Tal 3, ENE/UnB}
\membrodabancaiiifuncao{Examinador interno}
\membrodabancaiv{Prof. Fulano de Tal 4, EESC/USP}
\membrodabancaivfuncao{Examinador externo}
\membrodabancav{}
\membrodabancavfuncao{}
% -----------------------------------------------------------------------------------------------------

%line-numbers, inputencoding=utf8/latin1
%Define o estilo para listagens de código fonte
\lstset{
  numbers=left, %numeração de linhas à esquerda
  stepnumber=1,
  firstnumber=1,
  numberstyle=\tiny,
  extendedchars=true,
  frame=none,
  basicstyle=\footnotesize,
  stringstyle=\ttfamily,
  showstringspaces=false,
  %language=Java, %deve ser definida na inclusão de cada trecho de código, pois podem existir linguagens diferentes em exemplos diferentes
  breaklines=true,
  breakautoindent=true,
  %estilos de comentário de uma e várias linhas
  morecomment=[l]{--}, morecomment=[s]{/*}{*/}, morecomment=[s]{<!--}{-->}, morecomment=[s]{--[[}{--]]}
}

% Adição de metadados no PDF (propriedades do documento PDF)
\makeatletter
	 \hypersetup{
		 pdftoolbar=true,        % show Acrobat’s toolbar?
		 pdfmenubar=true,        % show Acrobat’s menu?
		 pdffitwindow=false,     % window fit to page when opened
		 pdfstartview={FitH},    % fits the width of the page to the window	 
		 pdftitle={\@title},
		 pdfauthor={\@author},
		 pdfsubject={\tipodemonografianome \ de\ \programastr \ em\ \areastr},   % subject of the document
		 pdfcreationdate={\pdfdate}
	 }
\makeatother


\makeindex
\makenomenclature %Necessário para gerar lista de siglas


\begin{document}

	\pdfbookmark[0]{Agradecimentos}{agradecimentos}
	\chapter*{Agradecimentos} \label{agradecimentos}
%* indica para nao adicionar numeracao ao titulo

Inclua seus agradecimentos aqui.

	\chapter{Resumo} \label{resumo}

Inclua o resumo aqui.

	
	\pdfbookmark[0]{Sumário}{sumario}
	\sumario
	
	\pdfbookmark[0]{Lista de Figuras}{listafiguras}
	\listadefiguras
	
	\pdfbookmark[0]{Lista de Tabelas}{listatabelas}
	\listadetabelas
	
	\pdfbookmark[0]{Lista de Códigos Fonte}{listacodigosfonte}
	\listadecodigosfonte
	
	\renewcommand{\nomname}{LISTA DE TERMOS E SIGLAS} %Define um caption à lista de siglas
	%Inclui a lista de siglas 
	\pdfbookmark[0]{Lista de Termos e Siglas}{nomenclatura}
	\printnomenclature[2.5cm] 
	
	\mainmatter %Inicia a numeracao normal cardinal
	\setcounter{page}{1} \pagenumbering{arabic} \pagestyle{plain}
	
	\include{introducao}
	\chapter{Arquitetura de comércio eletrônico para a TV Digital} \label{capitulo1}
%label cria um rótulo para o objeto, para permitir que ele seja referenciado com o comando \ref{nome-do-rotulo}


Neste capítulo é apresentada uma arquitetura para provimento de comércio eletrônico
para o Sistema Brasileiro de TV Digital (SBTVD) \nomenclature{SBTVD}{Sistema Brasileiro de TV Digital}. A mesma é uma arquitetura distribuída, baseada em componentes
reutilizáveis, os \textit{Web Services}, conhecida como Arquitetura Orientada a Serviços.

Segundo \cite{soares2007ginga}:

\begin{quote}
	Lorem ipsum dolor sit amet, consectetuer adipiscing elit. Ut purus elit, vesti- bulum ut, placerat ac, adipiscing vitae, felis. Curabitur dictum gravida mauris. Nam arcu libero, nonummy eget, consectetuer id, vulputate a, magna. 
\end{quote}

Desta forma, a arquitetura proposta foi definida, incluindo a implementação de um \textit{framework} de comunicação (baseado nos protocolos HTTP e SOAP) que é apresentado sucintamente neste capítulo, e em mais detalhes no Capítulo \ref{capitulo2}. Mais detalhes podem ser consultados em \cite{soares2007ginga}. Veja um exemplo na Listagem \ref{list:server}, que foi adaptada de \url{http://manoelcampos.com}.

Lorem ipsum dolor sit amet, consectetuer adipiscing elit. Ut purus elit, vestibulum ut, placerat ac, adipiscing vitae, felis. Curabitur dictum gravida mauris. Nam arcu libero, no- nummy eget, consectetuer id, vulputate a, magna. Donec vehicula augue eu neque. Pel- lentesque habitant morbi tristique senectus et netus et malesuada fames ac turpis egestas. Mauris ut leo. Cras viverra metus rhoncus sem. Nulla et lectus vestibulum urna fringilla ultrices. Phasellus eu tellus sit amet tortor gravida placerat. Integer sapien est, iaculis in, pretium quis, viverra ac, nunc. Praesent eget sem vel leo ultrices bibendum. Aenean fauci- bus. Morbi dolor nulla, malesuada eu, pulvinar at, mollis ac, nulla. Curabitur auctor semper nulla. Donec varius orci eget risus. Duis nibh mi, congue eu, accumsan eleifend, sagittis quis, diam. Duis eget orci sit amet orci dignissim rutrum.

Nulla malesuada porttitor diam. Donec felis erat, congue non, volutpat at, tincidunt tristi- que, libero. Vivamus viverra fermentum felis. Donec nonummy pellentesque ante. Phasellus adipiscing semper elit. Proin fermentum massa ac quam. Sed diam turpis, molestie vitae, placerat a, molestie nec, leo. Maecenas lacinia. Nam ipsum ligula, eleifend at, accumsan nec, suscipit a, ipsum. Morbi blandit ligula feugiat magna. Nunc eleifend consequat lorem. Sed lacinia nulla vitae enim. Pellentesque tincidunt purus vel magna. Integer non enim. Praesent euismod nunc eu purus. Donec bibendum quam in tellus. Nullam cursus pulvinar lectus. Donec et mi. Nam vulputate metus eu enim. Vestibulum pellentesque felis eu massa. 

\lstset{caption=Exemplo de aplicação servidora, label=list:server}
\begin{lstlisting}[language=C]
int main()
{ 
    FILE *fp;     int len;     
    static const int SIZE = 1024;
    struct sockaddr_in me, target;
    int sock=socket(AF_INET,SOCK_DGRAM,0);
    char arquivo[SIZE];
    me.sin_family=AF_INET;
    me.sin_addr.s_addr=htonl(INADDR_ANY); // endereco IP local 
    me.sin_port=htons(0); // porta local (0=auto assign)
    bind(sock,(struct sockaddr *)&me,sizeof(me));
    target.sin_family=AF_INET;
    target.sin_addr.s_addr=inet_addr("192.168.68.217"); // host local 
    target.sin_port=htons(8450); // porta de destino 

    if ((fp = fopen("video1.mp4","rb")) == NULL){
        printf("Arquivo nao pode ser aberto.\n"); return -1;
    }

    while(!feof(fp)) {
        len = fread(arquivo, 1, sizeof(arquivo), fp);
        sendto(sock,arquivo,sizeof(arquivo),0,(struct sockaddr *)&target,sizeof(target));
    }
    sendto(sock,"FIM",sizeof("FIM"),0,(struct sockaddr *)&target,sizeof(target));
    close(sock);
    return 0;
}
\end{lstlisting}

	\chapter{Protocolos de comunicação} \label{capitulo2}


\section{Uma associação}

% Gera 3 parágrafos com texto aleatório, apenas para exemplo. Apague o comando para remover tal texto.
\lipsum[1-3] 


\subsection{Motivos para abortar uma associação}

%lista não numerada
\begin{itemize}
	\item primeiro item;
	\item segundo item;
	\item terceiro item;
	\item quarto item.
\end{itemize}

\begin{figure}[h]
	\centering
	\includegraphics[scale=0.8]{imagens/red.png}
	\caption{Exemplo de imagem}
	\label{imagem1}
\end{figure}


Se durante o processo de configuração de uma associação for recebido como payload um hostname e esse hostname não puder ser resolvido em um tempo hábil deve se enviar um abort com a causa de erro de endereço não resolvido. Veja exemplo na Figura \ref{imagem1}.


% Gera 4 parágrafos com texto aleatório, apenas para exemplo. Apague o comando para remover tal texto.
\lipsum[1-4] 

A Tabela \ref{tabela1} a seguir apresenta os tipos de \textit{chunk} de um pacote do protocolo.

\begin{table}[ht!]
  \begin{center}
  \setlength{\belowcaptionskip}{10pt} % espao entre caption e tabela
  \footnotesize {
      \begin{tabular}{|p{4cm}|p{9cm}|}
	  \hline
	  \textbf{Nome} & \textbf{Função} \\
	  \hline
	  Iniciar & Usado para iniciar uma associação \\
	  \hline
	  Confirmacao & Segunda mensagem de uma configuração de uma associação\\
	  \hline
	  Mensagem & Terceira mensagem de uma configuração de uma associação\\
	  \hline
	  Cookie & Quarta mensagem de uma configuração de uma associação\\
	  \hline
	  Dados & Dados da aplicação\\
	  \hline
      \end{tabular}
  }
  \caption{Tipos de \textit{chunk} de um pacote SCTP}
  \label{tabela1}
  \end{center}
\end{table}
 
	\chapter*{Conclusão} \label{conclusao}
\addcontentsline{toc}{chapter}{\nameref{conclusao}}

% O * indica para nao adicionar numeracao ao titulo. 
% Neste caso, o capítulo não entra no sumário. O comando anterior força isso.

Insira sua conclusão aqui.
 
		
	%\bibliographystyle{abnt-num} % estilo bibliográfico ABNT numérico
	%\bibliographystyle{abnt-alf} % estilo bibliográfico ABNT alfabético
	\bibliographystyle{sbc}  % estilo bibliográfico da Sociedade Brasileira de Computação (SBC)
	
	%\renewcommand{\bibname}{REFERÊNCIAS BIBLIOGRÁFICAS} %Define o Caption da seção de bibliografia
	%\addcontentsline{toc}{chapter}{REFERÊNCIAS BIBLIOGRÁFICAS}
	
	%não pode ter espaço entre os nomes dos arquivos bib
	\bibliography{referencias}
	
	\chapter{Apêndice} \label{apendice}

Inclua os apêndices aqui.

\end{document}

