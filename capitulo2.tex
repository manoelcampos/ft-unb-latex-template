\chapter{Protocolos de comunicação} \label{capitulo2}


\section{Uma associação}

% Gera 3 parágrafos com texto aleatório, apenas para exemplo. Apague o comando para remover tal texto.
\lipsum[1-3] 


\subsection{Motivos para abortar uma associação}

%lista não numerada
\begin{itemize}
	\item primeiro item;
	\item segundo item;
	\item terceiro item;
	\item quarto item.
\end{itemize}

\begin{figure}[h]
	\centering
	\includegraphics[scale=0.8]{imagens/red.png}
	\caption{Exemplo de imagem}
	\label{imagem1}
\end{figure}


Se durante o processo de configuração de uma associação for recebido como payload um hostname e esse hostname não puder ser resolvido em um tempo hábil deve se enviar um abort com a causa de erro de endereço não resolvido. Veja exemplo na Figura \ref{imagem1}.


% Gera 4 parágrafos com texto aleatório, apenas para exemplo. Apague o comando para remover tal texto.
\lipsum[1-4] 

A Tabela \ref{tabela1} a seguir apresenta os tipos de \textit{chunk} de um pacote do protocolo.

\begin{table}[ht!]
  \begin{center}
  \setlength{\belowcaptionskip}{10pt} % espao entre caption e tabela
  \footnotesize {
      \begin{tabular}{|p{4cm}|p{9cm}|}
	  \hline
	  \textbf{Nome} & \textbf{Função} \\
	  \hline
	  Iniciar & Usado para iniciar uma associação \\
	  \hline
	  Confirmacao & Segunda mensagem de uma configuração de uma associação\\
	  \hline
	  Mensagem & Terceira mensagem de uma configuração de uma associação\\
	  \hline
	  Cookie & Quarta mensagem de uma configuração de uma associação\\
	  \hline
	  Dados & Dados da aplicação\\
	  \hline
      \end{tabular}
  }
  \caption{Tipos de \textit{chunk} de um pacote SCTP}
  \label{tabela1}
  \end{center}
\end{table}
